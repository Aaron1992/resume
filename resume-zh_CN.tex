% !TEX TS-program = xelatex
% !TEX encoding = UTF-8 Unicode
% !Mode:: "TeX:UTF-8"

\documentclass{resume}
\usepackage{zh_CN-Adobefonts_external} % Simplified Chinese Support using external fonts (./fonts/zh_CN-Adobe/)
%\usepackage{zh_CN-Adobefonts_internal} % Simplified Chinese Support using system fonts
\usepackage{linespacing_fix} % disable extra space before next section
\usepackage{cite}

\begin{document}
\pagenumbering{gobble} % suppress displaying page number

\name{常震}

\basicInfo{
  \email{chagnzhen850@gmail.com} \textperiodcentered\ 
  \phone{(+86) 186-5600-9986} \textperiodcentered\ 
   \linkedin[zmyth]{https://www.linkedin.com/in/zmyth/}}
 
\section{教育背景}
\datedsubsection{\textbf{中国科学技术大学}, 合肥}{2014.09 -- 2017.06}
\textit{硕士研究生}\ 计算机软件与理论
\datedsubsection{\textbf{中国科学技术大学}, 合肥}{2010.09 -- 2014.07}
\textit{学士}\ 大气科学

\section{工作/实习经历}

\datedsubsection{\textbf{上海华为技术有限公司}-上海}{2017.07--至今}
\role{软件工程师}{5G商用软件开发}
\begin{onehalfspacing}
\end{onehalfspacing}

\datedsubsection{\textbf{中国信息安全测评中心}-北京}{2015.08 -- 2016.05}
\role{实习}{符号执行研究,安全测评以及僵尸主机趋势地图开发}
\begin{onehalfspacing}

\end{onehalfspacing}

\section{项目经历}

\datedsubsection{\textbf{5G商用软件开发}}{2017.07--至今}
\role{C++}{工作项目}
\begin{onehalfspacing}
5G商用软件开发, C++。工作期间除完成本质工作外,多次作出改进效率的成果。获团队明日之星奖励。
\end{onehalfspacing}

\datedsubsection{\textbf{软件定义网络的路由算法}}{2016.03--2016.05}
\role{Python}{个人项目}
\begin{onehalfspacing}
在校期间参加的2016年华为软件精英挑战赛。设计路由算法并实现,解决一个类旅行商问题的NP-完全问题,要求实现较小的路径权值和较短的计算时间。使用语言C++。
\end{onehalfspacing}


\datedsubsection{\textbf{符号执行研究}}{2015.09--2015.11}
\role{C++}{实习项目}
\begin{onehalfspacing}
Windows平台符号执行工具Fuzzwin的研究和改造。
\end{onehalfspacing}

\datedsubsection{\textbf{僵尸主机趋势地图开发}}{2016.01--2016.02}
\role{Python}{实习项目}
\begin{onehalfspacing}
	开发威胁数据展示平台和收集程序。使用微软实时botnet数据,可视化展示威胁情报。
\end{onehalfspacing}

\datedsubsection{\textbf{德州扑克AI}}{2015.07--2015.09}
\role{Python}{个人项目}
\begin{onehalfspacing}
	在校期间参加的2015年华为软件精英挑战赛,利用Python语言开发的德州扑克AI, 使用了状态机和AI制作的基本原理。获上海安徽赛区12名。
\end{onehalfspacing}

\datedsubsection{\textbf{中科大评课社区}}{2015.04--2015.05}
\role{Python}{个人项目}
\begin{onehalfspacing}
在校期间和同学共同完成的兴趣作品。使用Python和Flask开发的中科大课程评论社区,负责后台视图部分的一部分逻辑开发。已开源在https://github.com/JING-TIME/ustc-course。
\end{onehalfspacing}

% Reference Test
%\datedsubsection{\textbf{Paper Title\cite{zaharia2012resilient}}}{May. 2015}
%An xxx optimized for xxx\cite{verma2015large}
%\begin{itemize}
%  \item main contribution
%\end{itemize}

\section{IT 技能}
% increase linespacing [parsep=0.5ex]
\begin{itemize}[parsep=0.5ex]
  \item 编程语言: Python > C++ > Java
  \item 平台: Linux
  \item 信息安全: 云计算安全,符号执行, web渗透,汇编
\end{itemize}

\section{获奖情况}
\datedline{\textit{第二名},2016年华为软件精英挑战赛上海安徽赛区}{}
%\datedline{其他奖项}{2015}

\section{其他}
% increase linespacing [parsep=0.5ex]
\begin{itemize}[parsep=0.5ex]
  \item 技术博客: https://zmyth.me
  \item GitHub: https://github.com/aaron1992
  \item 语言: 英语 - 熟练(CET6)
\end{itemize}

%% Reference
%\newpage
%\bibliographystyle{IEEETran}
%\bibliography{mycite}
\end{document}
